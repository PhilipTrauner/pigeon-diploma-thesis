\chapter{Introduction}

Re-usability of rockets is the future of the aerospace industry, especially in commercial applications such as space tourism and cargo transportation. In order to achieve this goal, traditional spacecraft design decisions have to be rethought to enable mass-production ready solutions that can return from space mostly intact. 

Few commercial aerospace companies have solved these problems to an extent that has has allowed them to capitalize on their spacecrafts. While interplanetary tourism is still a couple years away space transport services are in high demand.

Entering into this market segment is currently only feasible by investing a large sum of money on research and development in addition to financial cooperation with an already established space agency.

It is nigh impossible to attempt a market debut from within a non space faring nation and very difficult in those who already have successful space programs. Significantly cheaper solutions are required to prevent this new market from becoming a monopoly.

Solutions and systems developed by public research bodies like academic institutions could help mitigate the upfront cost.

\section{Task}

Launching non-reusable rockets is apart from the safety concerns considered a largely solved problem. Up until recently landing the boosters of these spacecraft was deemed unfeasible until propulsion based landing became possible through massive advancements in computing capabilities. 

This diploma thesis details creation of a propulsion based landing system by developing independent prototypes that test the utility of various designs and concepts.