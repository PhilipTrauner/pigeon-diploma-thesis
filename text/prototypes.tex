\chapter{Prototypes}

\section{Pigeon 9000}

\subsection{Concept}
The first prototype's name is alluding to Falcon 9 by SpaceX, the only reusable orbital class rocket with an propulsive landing booster currently available. 

The cheapest method to model a propulsions system was determined to be the usage of compressed air. 

To circumvent the implications of the Tsiolkovsky rocket equation it was decided against mounting a compressed air canister onto the actual rocket corpus and instead separating the air source and the model.

To regulate the generated thrust a controllable valve had to be fitted between the rocket corpus and the air source. 

The focus of this first prototype was gaining experiences handling thrust generated by compressed air systems. To minimize complexity, the movement of the corpus is constrained to the y-axis by a linear guide.

\subsection{Implementation}


\subsection{Manual control}


\subsection{Basic software control}


\subsection{Basic physical user interface}


\subsection{Improved physical user interface}


\subsection{Autonomous software control}