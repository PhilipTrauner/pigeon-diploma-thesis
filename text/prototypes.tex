\chapter{Prototypes}

\section{Pigeon 9000}

\subsection{Concept}
The first prototype's name is alluding to Falcon 9 \cite{falcon9} by SpaceX \cite{spacex}, the only reusable orbital class rocket with a propulsively landing booster currently available. 

The cheapest method to model a propulsion system was determined to be the usage of compressed air. 

To circumvent the implications of the Tsiolkovsky rocket equation \cite{rocket-equation} it was decided against mounting a compressed air canister onto the actual rocket corpus and instead separating the air source and the model.
 
% Explain effects of the Tsiolkovsky rocket equation

To regulate the generated thrust a controllable valve had to be fitted between the rocket corpus and the air source. 

The focus of this first prototype was gaining experience handling thrust generated by compressed air systems. To minimize complexity, the movement of the corpus is constrained to the y-axis by a linear guide.

\begin{figure}[h]
\centering

\includegraphics[width=60mm]{sketch_00_first_concept}

\caption{Draft of the first concept on a linear guide. Flexible tubes are fitted to the left and right side of the rocket corpus. Pressurized air is directed through these tubes.}
\end{figure}

   
\subsection{Implementation}
In the first meeting the most expensive and important part was acquired, a digitally controllable valve by Festo \cite{festo-valve} capable of handling pressures of up to 8bar. Additionally options for a linear guide and the rocket corpus were discussed. Using a table and utilizing it's legs as the guide and a PE pipe as the corpus was deemed the most practical approach. 

% Product details of Festo valve

Table leg diameters were gathered by browsing the table section of a large furniture store to compare to PE pipes available in nearby hardware stores. When suitable guides and corpora were found they were compared until a 37cm high side table \cite{table} with a leg diameter of 25mm and a pipe with 27mm inner diameter were picked. Additionally a special plastic glue, intended to be used to attach the pressure tubes to the rocket body, was purchased, even though the exact type of pressure tube that was going to be used was still unknown.

After the essential parts were gathered construction work began. A 5cm segment of the pipe was cut off to form the demonstrator corpus.

\begin{figure}[h]
\centering

\includegraphics[width=70mm]{sketch_01_table_and_corpus}

\caption{Table with corpus fitted to one of it's legs. The table legs have a diameter of 2,5cm and a height of 36cm. The tube that is used as the corpus is 5cm high, has an inside diameter of 2,7cm and an outside diameter of 3,2cm. Therefor the clearance between guide and corpus is 1mm.}
\end{figure}

% Insert math here

Initial weight measurements amounted to 8,13g. High tolerances were included in the initial calculations because attaching the pressure tubes to the body was expected to double it's weight, which was later confirmed by a measurement of 17,53g.

It was assumed that a nozzle at the end of the pressure tube was needed in order to produce a sufficient amount of thrust to lift the corpus.
To validate this assumption an electronic grain scale was used. Air pressure was applied to the scale surface from a distance of 1cm through a generic air blow gun with a nozzle diameter of 2mm. With 2,5bar out of 8bar available pressure, the scale was maxed out at 100g which led to the realization that the amount of thrust produced by the compressor was more than enough to achieve liftoff. 

After receiving a PU pressure tube with 6mm outer diameter and 4mm inner diameter, attempts were made to attach it to the corpus with the previously purchased plastic glue which was not able to connect the parts. In the course of investigating the problem it was discovered that the unreasonably expensive glue was not meant to stick to PE, which was mentioned in the fine print of the product packaging but went unnoticed. Instead, duct tape was used as a temporary solution. 

\begin{figure}[h]
\centering

\includegraphics[height=85mm]{sketch_03_corpus}

\caption{PE pipe used as rocket corpus. Duct tape is utilized to attach two PU tubes to either side of the body.}
\end{figure}

Consulting the documentation of the Festo valve revealed the need for a 24V power source. At this point in time it was still planned to power the whole prototype off battery power, for the sake of portability. The lack of cheap and long-lasting batteries that could be combined to provide 24V led to the conclusion, that a power supply was necessary. This resulted in an online purchase of a laptop charger \cite{power-supply} with the appropriate voltage from a common online retailer.

Connecting the PU tube to the Festo valve proved difficult because 4mm inner diameter is not an off-the-shelf size but instead used in industrial applications, therefor visits of every nearby hardware store with the intention of finding fitting connector parts were unsuccessful. Instead of ordering proper parts in a specialized online store, the decision was made to buy parts that were not intended for the tube diameter because long delivery times would have delayed the assembly by about a week. A crimp fitting with 6mm inner diameter, 10mm-16mm regular hose clamps and 11mm-13mm o-clip hose clamps were purchased. 

Connecting the crimp fitting to the pressure tube was successful with the help of a bunsen burner and a high amount of force. Electrical tape was wrapped around the tube to increase it's diameter in an effort to enable the clamps to fit.

The o-clip hose clamp could not be applied because it's diameter was still to large, even in it's crimped down state. The regular hose clamp fitted by over tightening it. This solution was not ideal because testing the connection revealed a high amount of pressure loss. It was therefor necessary to put up with the shipping delay and wait for the proper sized crimp fittings to arrive. A T-shaped crimp fitting was ordered in addition to regular crimp fittings and proper sized o-clip hose clamps in an effort to simplify the connection between the valve and both of the corpus tubes. Additionally the required electrical cable for the Festo valve was ordered because it too is not an off-the-shelf part and the risk generated by randomly sticking wires into the exposed ports on the valve was deemed too high.


\begin{figure}[h]
\centering

\includegraphics[width=60mm]{tube_botch}

\caption{Botched PU pressure tube fixed to an over sized crimp fitting by heating the tube ending and wrapping it in electrical tape to increase it's outer diameter. Additionally the connection is secured with an overtightened hose clamp.}
\end{figure}

The waiting period was utilized to find a fitting connector between the Festo valve and a customary compressor. This proved difficult because the connector size used by the valve is 1/8inch, which is mostly used in industrial applications. The compressor uses a more common 1/4inch connector, therefor an adapter was needed. Out of four nearby hardware stores only one carried any parts that were similar to the ones needed. These were promptly acquired and surprisingly fitted without any further issues.

The delivered T-shaped crimp fitting finally enabled a successful connection between the corpus and the valve without any noticeable pressure loss.  

\begin{figure}[h]
\centering

\includegraphics[width=110mm]{valve_assembly}

\caption{Festo valve with proper power connection. The T-shaped crimp fitting connects the valve to the pressure tubes that are attached to the corpus. The o-clip hose clamps provide additional protection against pressure loss. The compressor connector is attached to the valve via a 1/4inch to 1/8inch threaded adapter.}
\end{figure}

With most mechanical components in-place the attention was shifted to the electronic parts necessary, starting with a breadboard which was salvaged from an educational experimentation kit. 

To control the electronic valve with a microcomputer later on, it was plugged into the normal close port of a relay which allows for programmatic switching of 24V power. 

A Raspberry Pi 3 Model B \cite{raspberry-pi} was chosen as the development board because it enables high and low level programming with short iteration cycles thanks to a full-fledged underlying operating system \cite{raspbian} with networking capabilities. Direct memory access to the GPIO ports is well documented, I2C and SPI are supported with kernel command line flags and example code for most sensors is available on the internet. Additionally, a Raspberry Pi did not have to be acquired because one was already available.

An Arduino was decided against mainly because of the slower and less convenient development experience and the current lack of Rust \cite{rust} support, which is planned to be used as the projects low level language.

\begin{figure}[h]
\centering

\includegraphics[width=135mm]{gobnik_science}

\caption{First experiment to test out programmatic control of Festo valve with a Raspberry Pi 3 Model B and a relay.}
\end{figure}

Before mounting the components onto the table they were tested individually. Once the functionality of each part was validated a demo application was written in Rust to assess their combined functionality by switching the relay on and off. An audible click originated from the valve which gave the impression of a full switch operation happening inside the component.

All components were mounted onto the top of the table in addition to a distributor socket with the purpose of simplifying power delivery.


\begin{figure}[h]
\centering

\includegraphics[width=100mm]{sketch_04_topview}

\caption{Table construction from the top.}
\end{figure}

\begin{figure}[h]
\centering

\includegraphics[height=95mm]{sketch_05_sideview}

\caption{Table construction from the side.}
\end{figure}

\subsection{Manual control}

\subsection{Basic software control}


\subsection{Basic physical user interface}


\subsection{Improved physical user interface}


\subsection{Autonomous software control}