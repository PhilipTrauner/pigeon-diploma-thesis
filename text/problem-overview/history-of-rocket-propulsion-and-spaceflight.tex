\author{Sebastian Schaffler}
\section{History of Rocket Propulsion and Spaceflight}

\subsection{Early History}
The first time that a device, propelled by a stream of gaseous matter, was mentioned in literature, was in the writings of Aulus Gellius, a roman author and grammarian, written around 400 BC. This document states that Archytas, a Pythagorean of the same era, allegedly used a jet of steam to move a wooden object along wires. Any reported approaches on propulsion, which utilize Newtons 3rd Law, even though it has not even been defined as such until 1687 AC, were based on steam and so weak that they were barely able to move anything. Only the availability of black powder, which was an accidental discovery by the Chinese between 800 and 900 AC, was able to spark the idea of conceiving an apparatus that has the ability to propel itself forward freely. They where also the first to build rocket like devices which were actually able to fly, namely self propelled fire arrows. Self propelled pyrotechnics have been used as weapons and fireworks ever since. \cite{nasa-brief-history}

In the late 19th century Konstantin Tsiolkovsky, a Russian teacher and rocket scientist, had the idea of moving away from solid fuels and using combustible liquids. However he never actually built a device that utilized this approach. It was also him who formed a formula to calculate if a rocket would be capable of achieving the velocity required to allow for space travel, which would later become known as the famous Tsiolkovsky Rocket Equation. The equation itself had already been derived earlier by William Moore, a British scientist, but was not used in this context, because Tsiolkovsky was the first one to suggest space travel with the use of rockets.\cite{nasa-brief-history} \cite{rocket-equation-wiki}

\subsection{The Beginning of Rocket Science}

At first Tsiolkovskys solution did not gain much traction. This is the reason why Robert Goddard, an American physicist, had to derive the same equation without knowledge of Tsiolkovskys work, before he could become the first scientist who built a liquid fueled rocket capable of flying. About thirty years earlier a Peruvian inventor named Pedro Eleodoro Paulet supposedly built a liquid fueled rocket engine, generating about \SI{100}{\kilo\gram} of thrust, but did not manage to construct a craft that was able to achieve flight. The advances Goddard was able to achieve are considered base elements of every modern rocket propelled system. Among those achievements where the use of the convergent-diverged nozzle for rocketry, different methods of nozzle cooling, thrust vectoring and the throttleable liquid fuel engine.\cite{nasa-brief-history}

\begin{figure}[H]
\centering
\includegraphics[width=11cm]{goddard_rocket}
\caption{Goddards first liquid fueled rocket was a breakthrough in rocket development. It was fueled by liquid oxygen as well as gasoline and incorporated a number of newly developed technologies such as propellant turbines and a combustion chamber. Even tough it was victim to the pendulum rocket fallacy it was able to demonstrate an impressive first flight which carried the vehicle \SI{12.5}{\meter} high into the air and \SI{56}{\meter} downrange.}
\end{figure}

The early 1930s marked an upsurge in rocket science all around the world. The publication of a book on the use of liquid fuel by Herman Oberth, sparked especially high interest inside Germany. Hermann Oberth was a German scientist who is also considered an initiator of the field of modern rocket science along with Robert Goddard and Konstantin Tsiolkovsky. The German rocket building efforts where started by a team of hobbyist rocketeers that included the, later to become famous, rocket scientist Wernher von Braun. The same team was later recruited by the Nazis to develop the V2 weapon rocket in secrecy.\cite{nasa-brief-history}

Despite Goddards success in rocket development, rockets where largely seen as fictional devices in America, until the "Rocket Boys" became publicly known. They consisted of Caltech professor Theodore von Kármán and several graduate students of his. This group had the goal of developing and constructing rockets. Kármán named the team "Jet Propulsion Laboratory", abbreviated as JPL. At first they where just supported by their university, however soon the U.S. Army Air Corps took notice of JPL, who wanted small rocket powered assistance devices to help shorten the takeoff distance required by heavy aircraft. After they succeeded in this task, and the US entered the second world war, JPL started development of the first guided missiles. During the second world war they were also tasked with the technical analysis of rockets produced as part of the secret German V2 rocket development program. This led to the proposal of a research project aimed at duplicating the German designs. \cite{jpl-history-beginning}

\subsection{The Cold War Era}

In October 1957, The USSR announced that they were also actively conducting rocket related research. When they revealed their technology, the United States collectively realized, that their own technology was inferior. The USSR demonstrated their achievements with the first artificial satellite in orbit: Sputnik 1, a \SI{58}{\centi\meter} diameter metal ball, that had nothing but a radio transmitter on board. Driven by this demonstration of Russian capabilities, the Americans pooled their resources and botched together a rocket to lift the Explorer 1, a satellite previously constructed by JPL, into orbit. This rocket was Jupiter-C, a three stage vehicle of the Redstone rocket family for research and development. It was considered unable to insert a payload into a specific orbit, therefor a fourth stage was screwed on top which could insert the satellite into a stable orbit, thereby creating the Juno I launch vehicle. These two launches are considered the beginning of the "20th century space race".\cite{jpl-history-firstsat}\cite{nasa-beginning}

After Explorer 1 successfully went into orbit, the Americans saw the need of a federal space agency, thus creating the "National Aeronautics and Space Administration", abbreviated NASA. As NASA contracted large aerospace companies, the small non-profit JPL saw the need to alter their operating strategy. They managed to convince President Eisenhower and the U.S. Military to incorporate them into NASA, where they remain a pivotal research institution of the Agency to this day. \cite{jpl-history-joinnasa} As opposed to the centralized American program, the Soviet Union continued their space endeavors through multiple competing design bureaus. The Russian space program was also conducted much more secretively than its American counterpart, as it was strongly tied into the Soviet Military.\cite{ussr-space-program}

The first space race ended after sixteen years with the ten manned Saturn V flights of the Apollo Program, the last of which flew in December 1972. Of those ten missions four were research missions, unintended to land on the moon. Five of the six missions were designed to land, and did touch down on the Moons surface. Only one of those missions, namely Apollo 13 had a failure, however the crew managed to survive long enough, using the landing module as a lifeboat, to reach Earth alive. Despite being the mostly successful program, it did claim its victims, as the crew, intended to fly on the very first mission, Apollo 1, lost their lives during a cabin fire in a test capsule.\cite{apollo-mission}

\subsection{International Efforts}

Towards the end of the American lunar Program and after the tensions of the cold war started to relax, both superpowers, America and the USSR, agreed on a first joint venture, which was the Apollo-Soyuz Test Project, launched in July 1975. This project marked the start of international space cooperation, as its main objective was to dock a Russian Soyuz capsule to an American Apollo vehicle in orbit around earth. This mission also contributed a lot to the bettering of international relations.\cite{apollo-soyuz}

The time frame between 1965 and 1971 marked the emergence of space fairing capabilities in many nations around the globe. France launched its first satellite Astérix in November 1965. In 1970 Japan demonstrates its technological advancements with the successful launch of their Lambda 4 rocket and China placed their first satellite Dong Fang Hong-1 into orbit. The United Kingdom also achieved success with their Black Arrow rocket in 1971.\cite{fifty-years-sf}

In the very same year that the superpowers of the cold war carried out their joint mission, the European Space Agency, abbreviated as ESA, was established by 10 European nations. The first rocket operated by ESA, the French-built Ariane 1 was launched in 1979 from ESA's spaceport in French Guiana, a South American region of France.\cite{esa-history}

In 1977 NASA launched the two Voyager spacecraft, which went on a journey beyond the solar system. These two spacecraft are traveling away from our sun to this day and will continue to do so until they collide with something or are retrieved. Voyager 1 is the human made object furthest away from Earth. As of January 2018 Voyager 1 is \SI{141}{\astronomicalunit} or \SI{21e10}{\kilo\meter} away form the sun and moving at about \SI{3.6}{\astronomicalunit\per\year} or \SI{17.03}{\kilo\meter\per\second}.\cite{voyager}

The following decades were dominated with numerous advancements in terms of capability and safety, however the fact that launch vehicles and spacecraft are one-time use devices remained a constant throughout all these years. The American Space Shuttle was the first attempt of using the same hardware more than once, but sadly, even if it did succeed in many other tasks, it failed at making spaceflight easier, safer, more routine, and above all cheaper. This was mainly caused by the necessity of extensive inspection and refurbishment after every flight.\cite{space-shuttle-cost}

\subsection{The Commercial Space Sector}

Despite the Space Shuttle being unsuccessful in lowering the cost of spaceflight by moving away from singular use, the idea was not discarded. In the last decade multiple private entities discovered a viable business model in the development and flight of reusable space launch vehicles. Most notably, the two billionaires Elon Musk and Jeff Bezos took interest in this concept. Musk founded his company SpaceX in 2002. Bezos' developments with Blue Origin were not publicly known before their first test flight with their 'Goddard' Demonstrator in 2006. On this flight it reached an altitude of \SI{86.87}{\meter} and landed under engine power afterwards.\cite{bezos-goddard} In 2011, the same year as the Shuttles retirement, NASA awarded Draper Laboratory Inc. the task to develop a flight control system capable of flying and landing a rocket type vehicle, purely under engine power. Draper Laboratory Inc. then proceeded to contract Masten Space Systems for the construction of a demonstrator rocket. In February 2012 their Xombie demonstrator vehicle lifted \SI{164}{\meter} off the ground, moved \SI{164}{\meter} sideways, and landed again.\cite{xombie} 

SpaceX was not the first of the major spaceflight entities to develop a landing demonstrator. Their first test flight of 'Grasshopper' in September 2012 was not very impressive, as it literally just 'hopped' \SI{1.8}{\meter} in the air. However their development was the quickest and SpaceX became the first aerospace entity to implement the technology into a commercially viable vehicle. Their Falcon 9 vehicle had its maiden flight in 2010 and was continually upgraded over the following years. The technology developed with 'Grasshopper' and its follow up the 'F9R Dev' was also gradually phased in to the rocket. In April 2016 the Falcon 9 became the first orbital class rocket to ever make a controlled landing under its own engine power.\cite{spacex-wiki}

%The most recent development in the rocket and spaceflight scene is the appearance of commercial companies pursuing the goal of enabling space access for greatly reduced cost. Their reasoning behind the expected price reductions is in most cases based on moving away form the traditional one-time-use rocket and operate them more similar to airplanes, as in flying a payload out of the atmosphere and instead of discarding the huge and expensive first stage, they plan to fly them back, land, and fly again. This concept is generally been named reusability. Because of the expected emergence of commercial reusable rockets in the near future, governments that plan to stay viable as launch providers in the years to come, are driven to pour huge amounts of money into the development of more cost effective vehicles. This competition between the commercial and the governmental sectors of the space industry and also the race between the companies themselves, has sometimes been referred to as the "second space race" or the "commercial space race". Major players in this "race" are the governmental agencies NASA, ESA and Roscosmos and the companies United Launch Alliance, Space Exploration Technologies Corp. also called SpaceX and Blue Origin, LLC..

%The first vehicle capable of performing a rocket powered landing in addition to takeoff was the Masten Space Systems' Xombie suborbital rocket in February 2012. Since then, only SpaceX has managed to incorporate the ability to land into a full fledged, customer ready, launch system, the Falcon 9. SpaceX has developed the Falcon 9 from the ground up with reusability in mind and has tried and tested to land the rocket after as many successful space delivery missions as possible. The first time they managed to recover a flown booster stage was in December 2015, after ten years of intensive development, four unsuccessful landing attempts and three successful soft ocean touchdowns. As of early January 2018 SpaceX has recovered twenty Falcon 9 first stages and re flown five of them. In the future they expect to fly each vehicle at least ten times, possibly more often.

%Sources:
% http://www.esa.int/About_Us/Welcome_to_ESA/ESA_history/ELDO_ESRO_ESA_br_Key_dates_1960-2017

% https://www.nasa.gov/centers/dryden/news/NewsReleases/2012/12-06.html

% https://www.nasa.gov/centers/kennedy/about/history/timeline/2000s-decade.html

% https://www.grc.nasa.gov/www/k-12/TRC/Rockets/history_of_rockets.html

% https://spacexnow.com/

% https://www.space.com/4422-timeline-50-years-spaceflight.html



%Outline of what employees find important
