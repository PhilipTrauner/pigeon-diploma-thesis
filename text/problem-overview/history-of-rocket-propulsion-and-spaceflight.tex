\section{History of Rocket Propulsion and Spaceflight}
\author{Sebastian Schaffler}

\subsection{Early History}
The first time in literature a device propelled by a jet of matter, most reasonably in a gaseous state, was mentioned in the writings of Aulus Gellius, a roman author and grammarian, written around 400 BC. This document states that Archytas, a Greek Pythagorean of the same era, allegedly used a stream of steam to propel a object, made of wood, along wires. Any reported approaches on propulsion, which utilize Newtons 3rd Law, even though it has not even been defined as such until 1687 AC, where based on steam, and so weak that they were barely able to move anything. Only the availability of black powder, which was an accidental discovery by the Chinese between 800 and 900 AC, resulted in the idea of conceiving an apparatus that is capable of propelling itself forward freely, in other words a rocket. It was also them who built the first rocket like devices which were able to leave the ground, namely self propelled fire arrows.

Contrary to this, prior to the 13th century, the usage of pyrotechnical materials as self propelling projectiles was  widely believed to be infeasible, until the emergence of the "Book of Fires". This script, originally titled "Liber Ignium ad Combuerndos Hostes", depicts a range of incendiary weapons used since the eighth century, this includes two rocket type creations. The general knowledge about rocketry started to pick up slowly in the fifteenth century, with the first multi staged approaches being studied shortly after 1500 in Germany.

It was not until the late 19th century that Konstantin Tsiolkovsky, a Russian teacher and rocket scientist, had the idea of using fuel in liquid form, however he never applied the concept. It was also him who formed a formula to calculate if a rocket would be capable of achieving the velocity required to allow for space travel, which would later become known as the famous Tsiolkovsky Rocket Equation. The equation itself has already been derived earlier by William Moore, a British scientist, but was not used in the context of this problem. 

\subsection{The beginning of socket sience}

At first Tsiolkovskys solution did not receive a lot of attention in Russia and was not even published outside of the country. This is the reason why Robert Goddard, an American physicist, had to first derive this exact same equation independently, before he was able to become the first scientists to build a liquid fueled rocket capable of flying. Supposedly Pedro Eleodoro Paulet, a Peruvian inventor, built a liquid fueled rocket engine, generating about \SI{100}{\kilo\gram} of thrust, about thirty years earlier, but was never able to integrate it into a rocket to achieve flight. Goddards achievements in the early 19 hundreds are considered base elements of every modern rocket propelled system. Among them where the idea of using the De Laval nozzle, also known as convergent-diverged nozzle, for rocketry, different methods of nozzle cooling, thrust vectoring and the throttle able liquid fuel engine.

The early 1930s marked an upsurge in rocket science all around the world, especially in Germany, after Herman Oberth published a book in 1922 which proposes the use of liquid fuels. Hermann Oberth was a German scientist who is renowned as one of the initiators on the field of modern rocket science along with Robert Goddard and Konstantin Tsiolkovsky. The German rocket building efforts where started by a team of hobbyist rocketeers that included the, later to become famous, rocket scientist Wernher von Braun. The same team was later recruited by the Nazis to develop the V2 weapon rocket in secrecy.

Despite Goddards success in rocket development, rockets where largely seen as fictional devices in America, until the emergence of the "Rocket Boys" which consisted of Caltech professor Theodore von Kármán and several graduate students of his. This group has set it as their goal to develop and build Rockets. Kármán then named the team the "Jet Propulsion Laboratory", abbreviated and now widely known as the JPL. At first they had just the support of their university, but soon they drew the attention of the U.S. Army Air Corps who wanted small rocket powered assistance devices to help shorten the takeoff distance required by heavy aircraft. After they succeeded in this task, and the US entered the second world war, JPL started development of the first guided missiles. During the second world war they were also tasked with the technical analysis of rockets produced as part of the secret German V2 rocket development program. This led to the proposal of a research project aimed at duplicating the German designs.

\subsection{The Cold War era}

In October 1957, The USSR revealed that they were also actively conducting rocket related research. At the time of the announcement their technological progress was superior to that of the United States. They demonstrated their achievements with the first artificial satellite in orbit: Sputnik 1, a \SI{58}{\centi\meter} diameter metal ball with only a radio transmitter on board. Driven by this demonstration of Russian capabilities, the Americans pooled their resources and botched together a rocket capable of lifting the Explorer 1, a satellite previously constructed by JPL, into orbit. This Rocket was a Jupiter-C, a three stage research and development vehicle of the Redstone rocket family. This vehicle however did not have the capabilities to insert a payload into a specific orbit, therefore a fourth stage was screwed on top which could insert the satellite into a stable orbit, thereby creating the Juno I launch vehicle. These two launches marked the start of the 20th century space race.

After the launch of their first satellite, the Americans saw the need of a federal space agency, thus creating the National Aeronautics and Space Administration, abbreviated NASA. The creation of NASA and the entry of large aerospace companies into the rocketry business forced the small  non-profit Jet Propulsion Laboratory to alter their operating strategy. They managed to convince President Eisenhower and the Army to let them be part of NASA, where they still are one of the pivotal research institutions within the Agency. In contrast to the centralized program of the Americans the Soviet Union continued their space endeavors through multiple competing design bureaus, also the Russian space program was handled much more secretive than the American counterpart. Because of this secrecy very little information about their early days is public. But what is known, is that they managed their space endeavors very decentralized and many standalone design bureaus and manufacturers worked together.

The pinnacle and end of the sixteen year long first space race was marked by the ten manned Saturn V flights of the Apollo Program the last of which flew in December 1972. Of those ten missions four were research missions, not designed to land on the moon. Five of the six missions designed to land, managed to put a lunar module with two or three American astronauts on the Moons surface. Only one of those missions, namely Apollo 13 had a failure in orbit, but the crew was able to survive long enough, using the landing module as lifeboat, to return to Earth alive. Despite the  mostly successful program, it did claim its victims, as the crew, intended to fly on the very first mission Apollo 1, died in a cabin fire in a test capsule prior to launch. 

\subsection{International efforts}

During the last years of the American lunar Program and after the tensions of the cold war started to relax a little, both superpowers, America and the UDSSR agreed on a first joint project, the Apollo-Soyuz Test Project, launched in July 1975. This project marked the start of international space cooperation as its main purpose was to dock a Russian Soyuz capsule and an American Apollo capsule in orbit around earth. This mission also contributed a lot to the bettering of relations between the two nations.

The time frame between 1965 and 1971 marked the emergence of space fairing capabilities in many nations around the globe. France launched its first satellite Astérix in November 1965, in 1970 Japan demonstrates its technological advancements with the successful launch of their Lambda 4 rocket, in the same year China placed their first satellite Dong Fang Hong-1 into orbit and the United Kingdom achieved success with their Black Arrow rocket in 1971.

In the very same as the superpowers of the cold war carried out their joint mission, the European Space Agency, abbreviated as ESA, was established by 10 European nations including France and the UK. The first rocket, operated by ESA, the French-built Ariane 1, launched in 1979 form ESAs spaceport in French Guiana, a South American region of France.

In 1977 NASA launched the two Voyager spacecraft, which went on a journey throughout the solar system and beyond, these two spacecraft are to this day traveling away from our solar system and are expected to do so as long as they do not collide with anything or are retrieved. Voyager 1 is the furthest away human made object from Earth. As of January 2018 Voyager 1 is \SI{141}{\astronomicalunit} or \SI{21e10}{\kilo\meter} away form the sun and moving at about \SI{3.6}{\astronomicalunit\per\year} or \SI{17.03}{\kilo\meter\per\second}.

The following decades where dominated with numerous advancements in capability and safety of the huge launch vehicles, which are capable of accomplish awe inspiring missions. The most notable cornerstones in the numerous extraordinary feats achieved in the past 40 years are, the 135 successful launches of the Space Shuttle, which carries the title "Most Complex Machine Made By Man" for a reason and the construction of the International Space Station, a monument for what can be accomplished with international cooperation. The orbital laboratory was first visited by humans on November 2000, has had an uninterrupted human presence since then, has subsequently seen numerous upgrades and is expected to continue operations until 2025, when it is planned to be orbited due to decrepitude.

\subsection{The commertial space sector}

The most recent development in the rocket and spaceflight scene is the appearance of commercial companies pursuing the goal of enabling space access for greatly reduced cost. Their reasoning behind the expected price reductions is in most cases based on moving away form the traditional one-time-use rocket and operate them more similar to airplanes, as in flying a payload out of the atmosphere and instead of discarding the huge and expensive first stage, they plan to fly them back, land, and fly again. This concept is generally been named reusability. Because of the expected emergence of commercial reusable rockets in the near future, governments that plan to stay viable as launch providers in the years to come, are driven to pour huge amounts of money into the development of more cost effective vehicles. This competition between the commercial and the governmental sectors of the space industry and also the race between the companies themselves, has sometimes been referred to as the "second space race" or the "commercial space race". Major players in this "race" are the governmental agencies NASA, ESA and Roscosmos and the companies United Launch Alliance, Space Exploration Technologies Corp. also called SpaceX and Blue Origin, LLC..

The first vehicle capable of performing a rocket powered landing in addition to takeoff was the Masten Space Systems' Xombie suborbital rocket in February 2012. Since then, only SpaceX has managed to incorporate the ability to land into a full fledged, customer ready, launch system, the Falcon 9. SpaceX has developed the Falcon 9 from the ground up with reusability in mind and has tried and tested to land the rocket after as many successful space delivery missions as possible. The first time they managed to recover a flown booster stage was in December 2015, after ten years of intensive development, four unsuccessful landing attempts and three successful soft ocean touchdowns. As of early January 2018 SpaceX has recovered twenty Falcon 9 first stages and re flown five of them. In the future they expect to fly each vehicle at least ten times, possibly more often.

%Sources:
% http://www.esa.int/About_Us/Welcome_to_ESA/ESA_history/ELDO_ESRO_ESA_br_Key_dates_1960-2017

% https://www.nasa.gov/centers/dryden/news/NewsReleases/2012/12-06.html

% https://www.nasa.gov/centers/kennedy/about/history/timeline/2000s-decade.html

% https://www.grc.nasa.gov/www/k-12/TRC/Rockets/history_of_rockets.html

% https://spacexnow.com/

% https://www.space.com/4422-timeline-50-years-spaceflight.html

Sebastian S.
