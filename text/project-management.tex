\chapter{Project Management}
\author{Philip Trauner}
\section{Kanban}
Kanban is a process visualization technique inspired by the Toyota Production System, which was originally developed between 1948 and 1975 to address production shortcomings typically encountered by car manufacturers due to the scale of their operations. \\

The primary project management tool provided by Kanban is the Kanban board, on which cards are placed that represent tasks. It is split into columns through which cards gradually traverse as they are being worked on and completed. \\

Kanban was chosen as the project management solution for this diploma thesis because it does not dictate a perspective from which tasks have to be authored. Due to the experimental nature of this project it was assumed that the end-user focused feature descriptions typically found in user stories would prove a limiting factor in the effectiveness of project management approaches which heavily rely on them. Task interdependence can also be expressed more conveniently from a developers perspective than by a pure feature driven description. Additional, there is no need for daily meet-ups.

\section{Meetings}
Meetings with the project advisor were primarily held bi-monthly on Mondays and were used to discuss issues that arose during construction or programming.

\subsection{Results}
Both authors spent approximately \SI{340}{\hour} to complete this diploma thesis.

\begin{table}[H]
\centering
\begin{tabular}{llll}
\textbf{Name}       & \textbf{Programming} & \textbf{Hardware construction} & \textbf{Writing} \\
Philip Trauner      & \SI{115}{\hour} & \SI{102}{\hour} & \SI{120}{\hour} \\
Sebastian Schaffler & \SI{94}{\hour} & \SI{123}{\hour} & \SI{121}{\hour}            
\end{tabular}
\caption{Recorded working hours split into programming, hardware construction, and writing.}
\end{table}