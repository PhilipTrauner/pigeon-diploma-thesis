\section{Hardware implementation}
\subsection{Planning}
The dimensions of the second prototype had to be small enough to fit into the trunk of a car but also large enough to mitigate the stiffness of the tubes allowing for a movement range and therefor a guide height of about \SI{1}{\meter}. As a result of these constraints, fold-able side arms had to be considered for the sole purpose of suspending the pressure tubes and cables which had to be routed along the tubes to allow for a connection to the sensors. A \SI{80}{\centi\meter} by \SI{40}{\centi\meter} baseplate and a \SI{1}{\meter} high guide structure would form the basis of the main buildup.
% überlegen wie genau wir den test-stand und korpus umsetzen

To restrict the movement capabilities of the corpus to tilting and vertical positioning a rectangular or elliptical guide cross section was deemed to be best the option, however prevention of side-sliding movement was not possible with this solution. This was combated by splitting the guide into two parts and adding a restriction pin in the middle of the corpus. 

The most feasible construction approach was suspected to be two cylindrical rods separated by the width of the restriction pin. The structure inside the corpus, needed to achieve the desired movement restrictions, has to resemble a cuboid insertion slot for the guide wide enough to allow for tilting. To implement this shape it was decided to use two pins on the bottom and top of the corpus to model the slot shape and one vertically centered pin perpendicular to the long axis of the cuboid to function as the restriction pin. With this concept the geometry of the corpus does not matter as long as it is symmetrical and the movement restriction structure and sensors can fit inside.  

\subsection{Basic components}
Wood was chosen as the primary construction material for the test-stand because it can be processed with customary tools relatively quickly and easily in comparison to metals, which require specialized tools. A wooden cupboard top panel was purchased to be used as the baseplate. Cheap \SI{5}{\centi\meter} by \SI{3}{\centi\meter} slats were obtained as the construction material for the side arms. The first side-arm build attempt failed because the planks started twisting. Additionally, incorrect measurements lead to a high amount of waste and not enough remaining supply to redo the build. More expensive planks were then purchased to prevent future deformation. Measurements were respected more carefully the second time and the side arms could be manufactured successfully. Plywood sheets were obtained to serve as the top guide limiter. After the sheets were subjected to high humidity during storage they also started warping. Waste produced while working with the first slats was assembled into a T shape and used as the delimiter instead.

The same \SI{32}{\milli\meter} diameter PE pipes used as the corpus of the first prototype were chosen as the guides because of their low price and rather low friction when treated with silicon spray. A \SI{10}{\centi\meter} diameter pipe was chosen as the corpus because of its light weight and the ability to fit all required parts inside. 

A PE pipe with a diameter of \SI{15}{\milli\meter} was acquired as the building material for the cuboid movement restriction structure. After initial building efforts the results were deemed unsatisfactory because it was too wide and it was decided that another material was necessary. A carbon arrow with a diameter of \SI{5}{\milli\meter} was discovered while going through available supplies. The arrow shaft was able to be utilized effectively because of its light weight and small width.

unsatisfied 
unsatisfactory 

% schnell + billig = holzkonstruktion statt metall oder andere schwer verarbeitbare werkstoffe, 2 pe pipes als guides 1ner als stütze und future proofing falls mehr freiraum auf dem prototypen benötigt wird, große platte als grundplatte, große pipe für korpus + sehr dünne pipe für restriction im korpus

% skizze + beschreibung

% billigstmöglichstes material gekauft
% speerholzplatten und schlecht verarbeitete holz-latten
% beim zerschneiden der latten fehler aufgetreten, weil maße vor konstruktion nicht genau genug festgelegt
% hochwertigere latten umgestiegen, weil alten latten verbogen und splitter wurden als sicherheits-risiko eingestufe
% geschliffene latten gekauft
% speerholzplatten als begrenzung angedacht, wurden aber erst später verarbeitet
% hohe luftfeuchtugkeit in dev-hq -> nach 2 tagen vollkommen verbogen
% verschnitt von vorher ausreichend als begrenzung
% um pipes auf grundplatte zu mounten mehrere ideen:
% winkeln, stören aber den korpus
% zip-ties, als zu instabil bewertet, können noch immer stören
% holz-stöppel mit gleißen außenausmaß wie das innenausmaß der pe pipe
% mit hotglue zu befestigen
% durchbohrt um eine große feste schraube durchzubohren
% von unten durch die platte in die stöppel

% holz stöppel mit ähnlichen ausmaß wurden gefunden -> ca. 5mm clearance -> mit ducttape ausgeglichen (skizze)
% damit eine area ohne clearance genau passt und eine mit bisschen abstand wo kleber rein kann

% hinterer nur angeschraubt, da dieses mounting-verfahren sehr aufwendig ist, weil die stütze nicht primär als führung gedacht 
% festgeschraubt

% extreme caution is adviced wenn man mit viel hotglue arbeitet, da die masse zu schweren verbrennungen führen kann

% stöppel auch für deckel verwendet, da das prinzip sich als erfolgreich herausgestellt hat
% allerdings nicht angeklebt, weils durch die connection zur stütze nicht notwendig ist
% abnehmbar um den korpus wartbar zu machen

% dünne pipe noch immer zu dick für restrictions in korpus
% restrictions in korpus -> carbon pfeile

