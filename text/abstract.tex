\chapter{Abstract}

This diploma thesis depicts the development of two separate rocket lander prototypes that use compressed air for operation. These demonstrators are not standalone vehicles but part of test setups that provide them with propellant. 

The first prototype, called Pigeon 9000, is intended as a proof-of-concept. All sensors and the Raspberry Pi 3, which is used as the computing unit, are part of the stationary equipment. The actual flying corpus, which is moving along a guide in order to cut down complexity, only consists of one structural part and two pressure tubes. Both provide propellant and propel the device. The software of this prototype is designed in such a way that it provides the ability to rapidly prototype control algorithms. This has been achieved by implementing a daemon, called '9000d', which has direct control over critical hardware. This daemon is written in Rust, a language which grants the execution speed of a compiled language as well as memory safety at compile time. '9000d' accepts control instructions from programs written in Python, therefor enabling rapid prototyping.

Pigeon 9001, the second prototype, has the main purpose of being a testbed for various more sophisticated control algorithms, even exceeding the scope of this project. The software and hardware of Pigeon 9001 is built with many features that enable subsequent student teams to develop complex control algorithms, which can not be developed in one year without full time commitment.

Pigeon 9001 incorporates many lessons learned from Pigeon 9000, but also explores new techniques and features. The corpus of the second prototype also moves along a guide rail, but is not limited to simple linear motion anymore. The main computing unit, a Raspberry Pi Zero W, as well as a variety of sensors are moved onto the corpus. The back-end software of Pigeon 9001 is designed to utilize sensory data, provided by multiple devices inside a local network, via a producer-consumer model. The control architecture poses another major difference to Pigeon 9000, in that its daemon, now referred to as '9001d', which is also written in Rust, no longer takes control instructions from another piece of software. Instead it is designed to include all control algorithms in such a way that it either follows a preset flight profile or takes position parameters. The control algorithm developed in the scope of this project follows a preset flight profile. It detects when the corpus is being dropped and executes a soft landing sequence. Another control algorithm, able to hover at a certain altitude via pulse width modulation, has also been developed for testing purposes.